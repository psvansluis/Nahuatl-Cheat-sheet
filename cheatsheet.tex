%!TEX program = xelatex
\documentclass[12pt]{beamer}
\usetheme{Madrid}
\usetheme{Copenhagen}
% \usecolortheme{wolverine}

\usepackage[size=a4,scale=1.2]{beamerposter}
\usepackage{polyglossia}

\setbeamertemplate{headline}{}
\setbeamertemplate{footline}{}
\setbeamertemplate{frametitle}[default][center]

\setmainlanguage{english}
\usepackage{tikz}
\usepackage{fontspec}
\usepackage{booktabs}
\setlength{\tabcolsep}{.5em}
\usepackage{multirow}
\usepackage{bigdelim}
\usepackage{threeparttable}
\usefonttheme{serif}
\usepackage{noto}

\usepackage{multicol}
\setlength\multicolsep{0pt}
\setlength{\parskip}{5pt}

\makeatletter
\newcommand*{\textoverline}[1]{$\overline{\hbox{#1}}\m@th$}
\makeatother


\usepackage{lipsum}
\usepackage{forest}

\setlength{\leftmargini}{0.15cm}
\setlength{\leftmarginii}{0.25cm}

\newcommand{\nah}[1]{\textcolor{nahgrn}{#1}}
\newcommand{\trs}[1]{\textcolor{nahblu}{#1}}

\definecolor{nahred}{RGB}{224,32,13}
\definecolor{nahyel}{RGB}{148,114,16}
\definecolor{nahgrn}{RGB}{0,82,37}
\definecolor{nahblu}{RGB}{0,93,197}

\setbeamercolor{palette primary}{bg=nahred,fg=white}
\setbeamercolor{palette secondary}{bg=nahyel,fg=white}
\setbeamercolor{palette tertiary}{bg=nahgrn,fg=white}
\setbeamercolor{palette quaternary}{bg=nahblu,fg=white}
\setbeamercolor{structure}{fg=nahred} % itemize, enumerate, etc
\setbeamercolor{section in toc}{fg=nahred} % TOC sections


\usepackage[style=authortitle,isbn=false,backend=biber]{biblatex}
\addbibresource{nahuatlbib.bib}


\title{Nahuatl grammar}
% \subtitle{cheat sheet}

\begin{document}
\begin{frame}%
  {Classical Nahuatl grammar cheat sheet}{Phonology, pronouns, nouns and postpositions}
  \begin{columns}[t]
    \begin{column}{.23\linewidth}
      \begin{block}{Phonology \& standard orthography}
        \begin{threeparttable}
          \begin{tabular}{l@{\,}l@{\qquad}l@{\,}l@{\qquad}l@{\,}l@{\qquad}l@{\,}l}
            /a/  & \nah{a} & /j/  & \nah{y}                              & /o/  & \nah{o}                   & /t͡ɬ/ & \nah{tl}                             \\
            /aː/ & \nah{ā} & /k/  & \nah{c}, \nah{qu}\tnote{1}           & /oː/ & \nah{ō}                   & /t͡s/ & \nah{tz}                             \\
            /e/  & \nah{e} & /kʷ/ & \nah{cu}\tnote{2}, \nah{uc}\tnote{3} & /p/  & \nah{p}                   & /t͡ʃ/ & \nah{ch}                             \\
            /eː/ & \nah{ē} & /l/  & \nah{l}                              & /s/  & \nah{c}\tnote{1}, \nah{z} & /w/  & \nah{hu}\tnote{2}, \nah{uh}\tnote{3} \\
            /i/  & \nah{i} & /m/  & \nah{m}                              & /ʃ/  & \nah{x}                   & /ʔ/  & \nah{h}                              \\
            /iː/ & \nah{ī} & /n/  & \nah{n}                              & /t/  & \nah{t}                   &      &
          \end{tabular}
          \begin{tablenotes}
            \item[1] Before front vowels: \nah{e, ē, i, ī}
            \item[2] At beginning of syllables
            \item[3] At end of syllables
          \end{tablenotes}
        \end{threeparttable}
      \end{block}
      \begin{block}{Colonial orthography}
        \begin{itemize}
          \item Vowel length is often unwritten; when written, long vowels have a macron, and short vowels a breve.
                \begin{itemize}
                  \item  \nah{ātl} \trs{water}, \nah{ĭhui} \trs{thus}; or \nah{atl}, \nah{ihui}
                  \item  \nah{tepetl} > \nah{tepētl} \trs{mountain} or \nah{tēpetl} \trs{someone's mat}
                \end{itemize}
          \item  Long \nah{ō} is often written \nah{u}, but this is rare for short \nah{o}; \nah{v} for \nah{u} is common word-initially.
                \begin{itemize}
                  \item \nah{telpuchtli} > \nah{tēlpōchtli} \trs{young man}
                  \item \nah{vmpa} > \nah{ōmpa} \trs{yonder}
                \end{itemize}
          \item Vowels \nah{i}, \nah{ī} could also be written \nah{y} or \nah{j} and vice versa for semivowel \nah{y}.
                \begin{itemize}
                  \item \nah{yuh} > \nah{iuh} \trs{like}, \nah{injc} > \nah{in ic} \trs{so that}
                  \item \nah{yiollo} > \nah{īyōllo} \trs{his, her heart}
                \end{itemize}
          \item Glottal stop \nah{h} is usually only written between vowels; it can be indicated with  a grave or a circumflex accent on the preceding vowel.
                \begin{itemize}
                  \item \nah{amo} > \nah{ahmō} \trs{not}, \nah{ehecatl} > \nah{ehēcatl} \trs{wind}
                  \item \nah{tlàtoāni} > \nah{tlahtoāni} \trs{king}
                \end{itemize}
          \item Clusters \nah{oa} and \nah{ohua} are often confused, as are \nah{ia} and \nah{iya}.
                \begin{itemize}
                  \item \nah{cohuatl} > \nah{cōātl} \trs{snake}, \nah{pia} > \nah{piya} \trs{hold}
                \end{itemize}
          \item \nah{hu, uh} may be written \nah{o} or \nah{u, v}.
                \begin{itemize}
                  \item \nah{iehoatl} > \nah{yehhuātl} \trs{he, she, it}, \nah{eua} > \nah{ēhua} \trs{depart}
                \end{itemize}
          \item \nah{cua} is written \nah{qua, coa}; \nah{uc} can be written \nah{cuh, cu, uhc}.
                \begin{itemize}
                  \item \nah{tecuhtli} > \nah{tēuctli} \trs{lord}; \nah{quavitl} > \nah{cuahuitl} \trs{tree}
                \end{itemize}
          \item \nah{z} can be spelt \nah{ç}; \nah{tz} can be spelt \nah{tç}.
                \begin{itemize}
                  \item \nah{çoquitl} > \nah{zoquitl} \trs{clay}; \nah{tçonquiça} > \nah{tzonquīza} \trs{finish}
                \end{itemize}
          \item \nah{n} after vowels could be written with a tilde over the preceding vowel.
                \begin{itemize}
                  \item \nah{ipã} > \nah{īpan} \trs{on it};
                  \item \nah{ma tla\~qcã} > \nah{mā tlacuācān} \trs{let them eat}
                \end{itemize}
        \end{itemize}
      \end{block}
      \begin{block}{Scribal abbreviations}
        \vspace{-1.5ex}
        \begin{multicols}{2}
          \begin{itemize}
            \item \nah{a\textsuperscript{o}}: \nah{ahmō}
            \item \nah{-\textsuperscript{co}}: \nah{-tzinco}
            \item \nah{j\textsuperscript{o}}: \nah{Juan}
            \item \nah{\={q}}: \nah{que(h)}
            \item \nah{‍q̓}: \nah{qui(h)}
            \item \nah{\~{q}}: \nah{qua(h)}
            \item \nah{t(o)te\textsuperscript{o}}: \nah{totēcuiyō}
            \item \nah{\textoverline{xp}o, \textoverline{x}o}, etc.: \nah{Cristo}
          \end{itemize}
        \end{multicols}
      \end{block}
    \end{column}
    \begin{column}{.26\linewidth}
      \begin{block}{Sandhi}
        \begin{threeparttable}
          \begin{tabular}{l@{+ }l@{> }ll}
            \nah{uc}   & \nah{⁽ō⁾} & \nah{c⁽ō⁾}            & \nah{chiuc}+\nah{ōme} > \nah{chicōmē} \trs{7}                   \\
            \nah{tz}   & \nah{ch}  & \nah{chch}\tnote{1}   & \nah{nimitz}+\nah{chiya} > \nah{nimichiya} \trs{I await you}    \\
            \nah{ch}   & \nah{y}   & \nah{chch}\tnote{1}   & \nah{oquich}+\nah{yōtl} > \nah{oquichchōtl} \trs{manliness}     \\
            \nah{t}    & \nah{\#}  & \nah{h}               & \nah{mati}, pret.~\nah{mah} \trs{know}                          \\
            \nah{n, m} & \nah{hu}  & \nah{hu}              & \nah{am}+\nah{huālhuih} > \nah{ahuālhuih} \trs{y'all come}      \\
            \nah{l}    & \nah{tl}  & \nah{ll}              & \nah{pil}+\nah{tli} > \nah{pilli} \trs{child}                   \\
            \nah{l}    & \nah{y}   & \nah{ll}              & \nah{pil}+\nah{yōtl} > \nah{pillōtl} \trs{childhood}            \\
            \nah{uh}   & \nah{m}   & \nah{mm}              & \nah{cuauh}+\nah{māitl} > \nah{cuammāitl} \trs{tree-branch}     \\
            \nah{n}    & \nah{m}   & \nah{mm}              & \nah{on}+\nah{mati} > \nah{ommati} \trs{she feels inside}       \\
            \nah{n}    & \nah{p}   & \nah{mp}              & \nah{non}+\nah{pēhua} > \nah{nompēhua} \trs{I go forth}         \\
            \nah{m}    & \nah{\#}  & \nah{n}               & \nah{nemi}, pret.~\nah{nen} \trs{live}                          \\
            \nah{m}    & \nah{c}   & \nah{nc}              & \nah{cem}+\nah{cah} > \nah{cencah} \trs{very}                   \\
            \nah{m}    & \nah{ch}  & \nah{nch}             & \nah{quim}+\nah{chīhua} > \nah{quinchīhua} \trs{I do them}      \\
            \nah{m}    & \nah{cu}  & \nah{ncu}             & \nah{cem}+\nah{cuemitl} > \nah{cencuemitl} \trs{one field}      \\
            \nah{m}    & \nah{n}   & \nah{nn}              & \nah{am}+\nah{nemi} > \nah{annemi} \trs{y'all live}             \\
            \nah{m}    & \nah{t}   & \nah{nt}              & \nah{cem}+\nah{tetl} > \nah{centetl} \trs{one stone}            \\
            \nah{m}    & \nah{tl}  & \nah{ntl}             & \nah{am}+\nah{tlapiya} > \nah{antlapiya} \trs{y'all keep stuff} \\
            \nah{m}    & \nah{tz}  & \nah{ntz}             & \nah{cem}+\nah{tzontli} > \nah{centzontli} \trs{four hundred}   \\
            \nah{uh}   & \nah{p}   & \nah{pp}              & \nah{nāuh}+\nah{pa} > \nah{nāppa} \trs{four times}              \\
            \nah{ch}   & \nah{tz}  & \nah{tztz}\tnote{1}   & \nah{tōch}+\nah{tzintli} > \nah{tōtzintli} \trs{dear rabbit}    \\
            \nah{tz}   & \nah{y}   & \nah{tztz}\tnote{1}   & \nah{huitz}+\nah{yoh} > \nah{huitzoh} \trs{thorny}              \\
            \nah{n, m} & \nah{x}   & \nah{xx}\tnote{1}     & \nah{quim}+\nah{xōxa} > \nah{quixxōxa} \trs{he hexes them}      \\
            \nah{ch}   & \nah{x}   & \nah{xx}\tnote{1}     & \nah{nēch}+\nah{xōxa} > \nah{nēxxōxa} \trs{he hexes me}         \\
            \nah{x}    & \nah{y}   & \nah{xx}\tnote{1}     & \nah{mix}+\nah{yoh} > \nah{mixxoh} \trs{cloudy}                 \\
            \nah{y}    & \nah{\#}  & \nah{x, z}\tnote{2}{} & \nah{piya}, pret.~\nah{pix} \trs{do}                            \\
            \nah{y}    & \nah{C}   & \nah{xC}              & \nah{yēy}+\nah{pa} > \nah{yēxpa} \trs{three times}              \\
            \nah{h}    & \nah{V}   & \nah{yV}              & \nah{ah}+\nah{oc} > \nah{ayoc} \trs{no longer}                  \\
            \nah{z}    & \nah{y}   & \nah{zz}\tnote{1}     & \nah{ez}+\nah{yoh} > \nah{ezzoh} \trs{bloody}                   \\
            \nah{n, m} & \nah{z}   & \nah{zz}\tnote{1}     & \nah{zan}+\nah{cē} > \nah{zazcē} \trs{just one}                 \\
            \nah{ch}   & \nah{z}   & \nah{zz}\tnote{1}     & \nah{amēch}+\nah{zāloh} > \nah{amēzzāloh} \trs{he held y'all}   \\
          \end{tabular}%
          \begin{tablenotes}
            \item[1] Rarely written as geminates.
            \item[2] \nah{z} if it follows a \nah{z}, e.g.~\nah{celiya}, pret.~\nah{celiz} \trs{blossom}.

          \end{tablenotes}
        \end{threeparttable}
      \end{block}

      \begin{block}{Numerals}
        \vspace{-1.5ex}
        \begin{multicols}{2}
          \begin{itemize}
            \item \nah{cē}, \nah{cem-} \trs{1}
            \item \nah{ōme}, \nah{ōm-} \trs{2}
            \item \nah{(y)ē(y)i}, \nah{(y)ē(x)-} \trs{3}
            \item \nah{nāhui}, \nah{nāuh-} \trs{4}
            \item \nah{mācuilli}, \nah{mācuil-} \trs{5}
            \item \nah{chicuacē} \trs{6}
            \item \nah{chicōmē} \trs{7}
            \item \nah{chicuēyi} \trs{8}
            \item \nah{chiucnāhui} \trs{9}
            \item \nah{mahtlāctli} \trs{10}
            \item \nah{caxtōlli} \trs{15}
            \item \nah{(cem-)pōhualli} \trs{20}
            \item \nah{(cen-)tzontli} \trs{400}
            \item \nah{(cen-)xiquipilli} \trs{8000}
          \end{itemize}
        \end{multicols}
      \end{block}
      \begin{example}
        \nah{ētzontli īpan mahtlāctli onnāuhpōhualli ommahtlactli omōme} \trs{1492}
        \begin{center}
          \begin{tabular}{r@{~}c@{~}r@{~}c@{~}r@{~}r}
            \nah{ē} \trs{3}                                     & × & \nah{tzontli} \trs{400} & = & 1200 & \nah{īpan} \trs{+} \\
            (\nah{mahtlāctli} \trs{10} \nah{on} \trs{+} \nah{nāuh} \trs{4}) & × & \nah{pōhualli} \trs{20} & = & 280  & \nah{om} \trs{+}   \\
            (\nah{mahtlāctli} \trs{10} \nah{om} \trs{+} \nah{ōme} \trs{2})  & × & 1                 & = & 12   &              \\\hline
                                                          &   &                   &   & 1492 &              \\
          \end{tabular}
        \end{center}
      \end{example}
    \end{column}

    \begin{column}{.20\linewidth}
      \begin{block}{Derivational suffixes}
        \begin{threeparttable}
          \begin{itemize}
            \item \nah{-cauh}, pl.~\nah{-cahuān}  \trs{Vb-er}\tnote{1}
            \item \nah{-huia} \trs{use N, provide N}
            \item \nah{-(l)iztli} \trs{Vb-ing}
            \item \nah{-lli} \trs{something Vb-ed}
            \item \nah{-ni}, pl.~\nah{-nimeh}  \trs{Vb-er}\tnote{2}
            \item \nah{-qui}, pl.~\nah{-queh}  \trs{Vb-er}\tnote{2}
            \item \nah{-tia} \trs{provide with N, become N}
            \item \nah{-yo} inalienable possession
            \item \nah{-yoh} \trs{thing full of N}
            \item \nah{-yōtl} \trs{N-ness, N-ship}
          \end{itemize}
          \begin{tablenotes}
            \item[1] Only possessed nouns
            \item[2] Only absolutive nouns
          \end{tablenotes}
        \end{threeparttable}
      \end{block}
      \begin{example}
        \begin{itemize}
          \item \nah{miqui} \trs{die}, \nah{miquiliztli} \trs{death}
          \item \nah{nonac} \trs{my meat}, \nah{nonacayo} \trs{my flesh}
          \item \nah{tlamati} \trs{he knows things}, \nah{tlamatini} \trs{(he is a) scholar}
        \end{itemize}
      \end{example}
      \begin{block}{Numeral classifiers}
        \begin{itemize}
          \item \nah{-tetl} \trs{rock}: small things
          \item \nah{-pāntli} \trs{ridge, furrow}: rows
          \item \nah{-tlamantli} \trs{thing}: foldable, stackable things
          \item \nah{-ōlōtl} \trs{maize cob}: round, oblong things
        \end{itemize}
      \end{block}
      \begin{block}{Distributive numbers}
        Distributive numbers are formed by reduplication of the first syllable, usually in \nah{C\=V-}:
        \begin{itemize}
          \item \nah{cēcen} \trs{each one, one each}
          \item \nah{māmācuīlli} \trs{by fives, five each}
        \end{itemize}
      \end{block}
    \end{column}

    \begin{column}{.23\linewidth}
      \begin{block}{Possessive prefixes}
        \begin{tabular}[t]{lll}
          \multicolumn{3}{c}{Specific possessor} \\
            & sg.       & pl.                    \\
          1 & \nah{no-} & \nah{to-}              \\
          2 & \nah{mo-} & \nah{amo-}             \\
          3 & \nah{ī-}  & \nah{īn-/īm-}          \\
        \end{tabular}
        \qquad
        \begin{tabular}[t]{ll}
          \multicolumn{2}{c}{Indefinite possessor} \\
          \nah{tē-}  & \trs{someone's}             \\
          \nah{tla-} & \trs{something's}
        \end{tabular}
      \end{block}

      \begin{block}{Absolutive/possessive \& number suffixes}
        \begin{itemize}
          \item
                \begin{tabular}[t]{ll}
                  \multicolumn{2}{c}{Absolutive} \\
                  sg.         & pl.              \\
                  \nah{C-tli} & \nah{-tin, -meh} \\
                  \nah{V-tl}  & \nah{-(me)h}     \\
                  \nah{l-li}  & \nah{-tin}       \\
                  \nah{-in}   & \nah{-meh, -tin} \\
                \end{tabular}%
                \qquad
                \begin{tabular}[t]{ll}
                  \multicolumn{2}{c}{Possessive }                                   \\
                  sg.         & pl.                                                 \\
                  \nah{C-hui} & \multicolumn{1}{l}{\multirow{2}[0]{*}{\nah{-huān}}} \\
                  \nah{V-uh}  &                                                     \\
                \end{tabular}%
          \item Absolutive: non-possessed nouns.
          \item Possessive: nouns with a possessive prefix.
          \item Plurals in \nah{-h} and \nah{-tin} may have reduplication of the first syllable of the stem, whose vowel is lengthened:
                \begin{itemize}
                  \item \nah{conētl} \trs{child}, pl.~\nah{cōconeh}
                  \item \nah{tōchtli} \trs{rabbit}, pl.~\nah{tōtōchtin}
                  \item \nah{cōātl} \trs{snake}, pl.~\nah{cōcōah}
                \end{itemize}
          \item Only animate nouns inflect for number:
                \begin{itemize}
                  \item \nah{cihuātl} \trs{woman}, \nah{cihuah} \trs{women}
                  \item \nah{tetl} \trs{rock, rocks}
                \end{itemize}
        \end{itemize}
      \end{block}
      \begin{block}{Postpositions}
        \begin{tabular}{ll}
          \nah{-c(o)}     & \trs{in, at}                         \\
          \nah{-ca}       & \trs{by means of, through, with}     \\
          \nah{-huān}     & \trs{(together) with, moreover, and} \\
          \nah{-īcampa}   & \trs{behind}                         \\
          \nah{-ihtic}    & \trs{inside, within}                 \\
          \nah{-īxpan}    & \trs{in front of, facing}            \\
          \nah{-nāhuac}   & \trs{next to, close to}              \\
          \nah{-pampa}    & \trs{because of, concerning}         \\
          \nah{-pan}      & \trs{on, in, at}                     \\
          \nah{-(t)icpac} & \trs{on top of}                      \\
          \nah{-tech}     & \trs{next to, adhering to}           \\
          \nah{-tlan}     & \trs{by, next to, below}             \\
          \nah{-tzīntlan} & \trs{below, underneath}              \\
        \end{tabular}%
        \begin{itemize}
          \item Postpositions are suffixed to posessive pronouns and nouns:
                \begin{itemize}
                  \item \nah{notech} \trs{next to me}
                  \item \nah{īhuan} \trs{with it}
                  \item \nah{tlalticpac} \trs{on the earth}
                \end{itemize}
          \item N-Postposition has a general indefinite meaning; \nah{ī/īm}-Postposition \nah{in} N has a specific definite meaning:
                \begin{itemize}
                  \item \nah{tepan} \trs{on stone(s)}
                  \item \nah{īpan in tetl} \trs{on the stone(s)}
                \end{itemize}
          \item Postpositions may take the reverential suffix \nah{-tzinco}
        \end{itemize}
      \end{block}
    \end{column}
  \end{columns}
\end{frame}

\begin{frame}%
  {Classical Nahuatl grammar cheat sheet}{The verbal complex}
  \begin{columns}[t]
    \begin{column}{0.18\linewidth}
      \begin{block}{Verb prefixes}
        \begin{tabular}[t]{ll}
          \multicolumn{2}{l}{Imperative/optative marker} \\
          \nah{mā}        & \trs{if, let it be}          \\
          \nah{tlā}       & \trs{if, let it be (please)} \\
          \nah{māca[mō]}  & \nah{mā + ahmō}              \\
          \nah{tlāca[mō]} & \nah{tlā + ahmō}             \\
        \end{tabular}
        \\[1ex]
        \begin{tabular}[t]{ll}
          \multicolumn{2}{l}{Negative marker} \\
          \nah{ah}   & \trs{not, un-}         \\
          \nah{ahmō} & \trs{not, no}          \\
        \end{tabular}
        \\[1ex]
        \begin{tabular}[t]{ll}
          \multicolumn{2}{l}{Antecessive prefix} \\
          \nah{ō} & \trs{already}                \\
        \end{tabular}
        \\[1ex]
        \begin{threeparttable}
          \begin{tabular}[t]{lll}
            \multicolumn{3}{l}{Subject pronoun}             \\
              & sg.                & pl.                    \\
            1 & \nah{ni-}          & \nah{ti-}              \\
            2 & \nah{ti-}\tnote{1} & \nah{am-/an-}\tnote{1} \\
            3 & \nah{ø-}           & \nah{ø-}               \\
          \end{tabular}
          \begin{tablenotes}
            \item[1] \nah{xi-} if optative
          \end{tablenotes}
        \end{threeparttable}
        \\[1ex]
        \begin{tabular}[t]{llll}
          \multicolumn{3}{l}{Definitive object pronoun} \\
            & sg.           & pl.                       \\
          1 & \nah{nēch-}   & \nah{tēch-}               \\
          2 & \nah{mitz-}   & \nah{amēch-}              \\
          3 & \nah{c-/qui-} & \nah{quim-/im-}           \\
        \end{tabular}
        \\[1ex]
        \begin{tabular}[t]{ll}
          \multicolumn{2}{l}{Directional marker} \\
          \nah{huāl} & \trs{hither}              \\
          \nah{on}   & \trs{thither}             \\
        \end{tabular}
        \\[1ex]
        \begin{tabular}[t]{llll}
          \multicolumn{4}{l}{Reflexive pronoun}                                                              \\
            & sg.                           & pl.       &                                                    \\
          1 & \nah{no-}                     & \nah{to-} & \trs{\hspace{-1em}\rdelim\}{3}{*}[-self, -selves]} \\
          2 & \multicolumn{2}{c}{\nah{mo-}}                                                                  \\
          3 & \multicolumn{2}{c}{\nah{mo-}}                                                                  \\
        \end{tabular}
        \\[1ex]
        \begin{tabular}[t]{ll}
          \multicolumn{2}{l}{Indefinite object pronoun} \\
          \nah{tē-}  & \trs{someone, people}            \\
          \nah{tla-} & \trs{something, stuff}           \\
        \end{tabular}
      \end{block}
      \begin{block}{Stem reduplication}
        The first syllable of a verb stem is reduplicated with:
        \begin{itemize}
          \item \nah{C\=V-} to make a frequentative,
          \item \nah{CVh-} to make a distributive.
        \end{itemize}
      \end{block}
      \begin{example}
        \begin{itemize}
        \item \nah{chōca} \trs{cries}: \nah{chōchōca} \trs{cries regularly}
        \item \nah{pāqui} \trs{is glad}: \nah{pahpāqui} \trs{is glad (on various occasions)}
        \end{itemize}
      \end{example}
    \end{column}
    \begin{column}{.18\linewidth}
      \begin{block}{Verb classes}
        \begin{tabular}{ll}
          Cl 1 & Vbs in \nah{-VCCV}           \\
          Cl 2 & Vbs in \nah{-VCV}            \\
          Cl 3 & Vbs in \nah{-VV}             \\
          Cl 4 & One-syllable Vbs in \nah{-a}
        \end{tabular}\\
        Exceptions
        \begin{itemize}
          \item Vbs in \nah{-Co}, \nah{-tla}, \nah{-ca}: Cl 1
          \item Intransitive Vbs in \nah{-hua}: Cl 1
          \item One-syllable Vbs in \nah{-i}: Cl 1
          \item Transitive Vbs in \nah{-hua}: Cl 2
          \item Vbs in \nah{-ya}: Cl 1 or 2
          \item \nah{tōna} \trs{be warm}, \nah{pāca} \trs{wash}: Cl 1
          \item \nah{zōma} \trs{become angry}: Cl 4
        \end{itemize}
      \end{block}
      \begin{block}{Verbal bases}
        B~1 is the dictionary form. Other bases are derived as follows:
        \begin{tabular}{lll}
               & B~2           & B~3                \\
          Cl 1 & B~2 = B~1     & B~3 = B~1          \\
          Cl 2 & \nah{-V > -ø} & B~3 = B~1          \\
          Cl 3 & \nah{-V > -h} & \nah{-V₁V₂ > -V₁ː} \\
          Cl 4 & \nah{-h}      & \nah{-V > -Vː}
        \end{tabular}
      \end{block}
      \begin{example}
        \begin{tabular}{lllll}
               & B~1          & B~2          & B~3          &             \\
          Cl 1 & \nah{chōca-} & \nah{chōca-} & \nah{chōca-} & \trs{cry}   \\
          Cl 2 & \nah{yōli-}  & \nah{yōl-}   & \nah{yōli-}  & \trs{live}  \\
          Cl 3 & \nah{āltia-} & \nah{āltih-} & \nah{āltī-}  & \trs{bathe} \\
          Cl 4 & \nah{cua-}   & \nah{cuah-}  & \nah{cuā-}   & \trs{eat}
        \end{tabular}
      \end{example}
      \begin{block}{Tense/Mood endings}
        \begin{threeparttable}
          \begin{tabular}{llll}
            Tense/Mood  & B & sg.                & pl.                    \\
            Present     & 1 & \nah{-ø}           & \nah{-h}               \\
            Habitual    & 1 & \nah{-ni}\tnote{1} & \nah{-nih}\tnote{1}    \\
            Imperfect   & 1 & \nah{-ya}\tnote{2} & \nah{-yah}             \\
            Preterite   & 2 & \nah{-c/ø}\tnote{3}  & \nah{-queh}            \\
            Pluperfect  & 2 & \nah{-ca}          & \nah{-cah}             \\
            Admonitive  & 2 & \nah{-h}\tnote{3}  & \nah{-(h)tin}\tnote{3} \\
            Future      & 3 & \nah{-z}           & \nah{-zqueh}           \\
            Optative    & 3 & \nah{-ø}           & \nah{-cān}             \\
            Conditional & 3 & \nah{-zquiya}      & \nah{-zquiyah}
          \end{tabular}
          \begin{tablenotes}
            \item[1] Preceding V lengthened
            \item[2] Preceding V lengthened except Cl 1
            \item[3] Cl 1: \nah{-c}, otherwise \nah{-ø}
          \end{tablenotes}
        \end{threeparttable}
      \end{block}
    \end{column}
    \begin{column}{.18\linewidth}
      \begin{block}{Irregular verbs}
        \begin{tabular}[t]{lll}
          \multicolumn{3}{l}{\nah{cā/ye} \trs{be}} \\
                & sg.         & pl.                \\
          Pres. & \nah{cah}   & \nah{cateh}        \\
          Impf. & \nah{yeya}  & \nah{yeyah}        \\
          Pret. & \nah{catca} & \nah{catcah}       \\
          Fut.  & \nah{yez}   & \nah{yezqueh}      \\
        \end{tabular}%
        \\[1ex]
        \begin{tabular}[t]{lll}
          \multicolumn{3}{l}{\nah{huītza} \trs{go}} \\
                & sg.          & pl.                \\
          Pres. & \nah{huītz}  & \nah{huītzeh}      \\
          Impf. & \nah{huītza} & \nah{huītzah}      \\
        \end{tabular}%
        \\[1ex]
        \begin{tabular}[t]{lll}
          \multicolumn{3}{l}{\nah{yā/huih} \trs{come}} \\
                & sg.        & pl.                     \\
          Pres. & \nah{yauh} & \nah{huih}              \\
          Impf. & \nah{yāya} & \nah{yāyah}             \\
          Pret. & \nah{yah}  & \nah{yahqueh}           \\
          Fut.  & \nah{yāz}  & \nah{yāzqueh}           \\
        \end{tabular}%
        \\[1ex]
        \begin{tabular}[t]{lll}
          \multicolumn{3}{l}{\nah{huāllā/huālhuih} \trs{come}} \\
                & sg.             & pl.                        \\
          Pres. & \nah{huāllauh}  & \nah{huālhuih}             \\
          Impf. & \nah{huālhuiya} & \nah{huālhuiyah}           \\
          Pret. & \nah{huāllah}   & \nah{huāllahqueh}          \\
          Fut.  & \nah{huāllaz}   & \nah{huāllazqueh}          \\
        \end{tabular}%
      \end{block}
    \end{column}
    \begin{column}{.18\linewidth}
      \begin{block}{Auxiliary verbs}
        Auxiliary verbs specify physical position of the main verb, condition under which the main verb takes place and mark aspect, and are made as follows:
        \begin{itemize}
          \item The main verb in the preterite stem
          \item Ligature morpheme \nah{-t(i)-}
          \item The auxiliary verb, bearing tense and number
        \end{itemize}
        \begin{threeparttable}
          \begin{tabular}{lll}
            \multicolumn{1}{c}{Verb} & \multicolumn{2}{c}{Meaning}                            \\
                                     & \multicolumn{1}{c}{Ind.}    & \multicolumn{1}{c}{Aux.} \\
            \nah{cah}                & \trs{be}                    & \trs{be Vb-ing}          \\
            \nah{ēhua}               & \trs{rise}                  & \trs{depart   Vb-ing}    \\
            \nah{huetzi}             & \trs{fall}                  & \trs{Vb   quickly}       \\
            \nah{huītz}              & \trs{come}                  & \trs{come   Vb-ing}      \\
            \nah{ihcac}              & \trs{stand}                 & \trs{stand   Vb-ing}     \\
            \nah{mani}               & \trs{be, cover}             & \trs{be Vb-ing}          \\
            \nah{nemi}               & \trs{live}                  & \trs{go   about Vb-ing}  \\
            \nah{(on)oc}             & \trs{lie}                   & \trs{lie   Vb-ing}       \\
            \nah{quīza}              & \trs{emerge}                & \trs{pass   Vb-ing}      \\
            \nah{yauh}\tnote{1}      & \trs{go}                    & \trs{be Vb-ing}          \\
          \end{tabular}
          \begin{tablenotes}
            \item[1] \nah{ti+yauh} > \nah{-tiuh}
          \end{tablenotes}
        \end{threeparttable}

      \end{block}
      \begin{block}{Verbs of purposive motion}
        Purposive motion suffixes take the present stem
        \begin{tabular}{llll}
                                       &     & \trs{Come}     & \trs{Go}          \\
          \multirow{2}{*}{Pres./Pret.} & sg. & \nah{-co}      & \nah{-to}         \\
                                       & pl. & \nah{-coh}     & \nah{-toh}        \\
          \multirow{2}{*}{Future}      & sg. & \nah{-quiuh}   & \nah{-tīuh}       \\
                                       & pl. & \nah{-quihuih} & \nah{-tīhuih}     \\
          \multirow{2}{*}{Optative}    & sg. & \nah{-qui}     & \nah{-h,   -ti}   \\
                                       & pl. & \nah{-quih}    & \nah{-tih,   tin} \\
        \end{tabular}
      \end{block}
      \begin{block}{Verb suffixes}
        \begin{tabular}[t]{ll}
          \multicolumn{2}{l}{Causative}     \\
          \nah{-(l)tia} & \trs{cause to Vb}
        \end{tabular}
        \\[1ex]
        \begin{tabular}[t]{ll}
          \multicolumn{2}{l}{Passive}                                  \\
          \nah{-(l)o}   & \trs{\hspace{-1em}\rdelim\}{3}{*}[be Vb'ed]} \\
          \nah{-(o)hua} &                                              \\
          \nah{-hualo}  &                                              \\
        \end{tabular}
        \\[1ex]
        \begin{tabular}[t]{ll}
          \multicolumn{2}{l}{Applicative}         \\
          \nah{-i(l)ia} & \trs{do Vb for someone}
        \end{tabular}
        \\[1ex]
        A reflexive prefix combined with either a causative or an applicative suffix creates a reverential verb
      \end{block}

    \end{column}
    \begin{column}{.18\linewidth}
      \begin{example}
        \nah{niquittato} \trs{I went to see it}

      \end{example}
      \begin{block}{Literature consulted}
        \nocite{lockhart_NahuatlWrittenLessons2001,andrews_ClaNahuatl03,jordan_BriefNotesNahuatl,canger_ReduplicationNahuatlDialectal1981,wood_OnlineNahuatlDictionary}
        \printbibliography
      \end{block}
    \end{column}
  \end{columns}
  \vfill
\end{frame}
\end{document}
