\begin{frame}%
{Classical Nahuatl grammar cheat sheet}{The verbal complex}
\begin{columns}[t]
    \begin{column}{0.18\linewidth}
    \begin{block}{Verb prefixes}
        \begin{tabular}[t]{ll}
        \multicolumn{2}{l}{Imperative/optative marker} \\
        \nah{mā}        & \trs{if, let it be}          \\
        \nah{tlā}       & \trs{if, let it be (please)} \\
        \nah{māca[mō]}  & \nah{mā + ahmō}              \\
        \nah{tlāca[mō]} & \nah{tlā + ahmō}             \\
        \end{tabular}
        \\[1ex]
        \begin{tabular}[t]{ll}
        \multicolumn{2}{l}{Negative marker} \\
        \nah{ah}   & \trs{not, un-}         \\
        \nah{ahmō} & \trs{not, no}          \\
        \end{tabular}
        \\[1ex]
        \begin{tabular}[t]{ll}
        \multicolumn{2}{l}{Antecessive prefix} \\
        \nah{ō} & \trs{already}                \\
        \end{tabular}
        \\[1ex]
        \begin{threeparttable}
        \begin{tabular}[t]{lll}
            \multicolumn{3}{l}{Subject pronoun}             \\
            & sg.                & pl.                    \\
            1 & \nah{ni-}          & \nah{ti-}              \\
            2 & \nah{ti-}\tnote{1} & \nah{am-/an-}\tnote{1} \\
            3 & \nah{ø-}           & \nah{ø-}               \\
        \end{tabular}
        \begin{tablenotes}
            \item[1] \nah{xi-} if optative
        \end{tablenotes}
        \end{threeparttable}
        \\[1ex]
        \begin{tabular}[t]{llll}
        \multicolumn{3}{l}{Definitive object pronoun} \\
            & sg.           & pl.                       \\
        1 & \nah{nēch-}   & \nah{tēch-}               \\
        2 & \nah{mitz-}   & \nah{amēch-}              \\
        3 & \nah{c-/qui-} & \nah{quim-/im-}           \\
        \end{tabular}
        \\[1ex]
        \begin{tabular}[t]{ll}
        \multicolumn{2}{l}{Directional marker} \\
        \nah{huāl} & \trs{hither}              \\
        \nah{on}   & \trs{thither}             \\
        \end{tabular}
        \\[1ex]
        \begin{tabular}[t]{llll}
        \multicolumn{4}{l}{Reflexive pronoun}                                                              \\
            & sg.                           & pl.       &                                                    \\
        1 & \nah{no-}                     & \nah{to-} & \trs{\hspace{-1em}\rdelim\}{3}{*}[-self, -selves]} \\
        2 & \multicolumn{2}{c}{\nah{mo-}}                                                                  \\
        3 & \multicolumn{2}{c}{\nah{mo-}}                                                                  \\
        \end{tabular}
        \\[1ex]
        \begin{tabular}[t]{ll}
        \multicolumn{2}{l}{Indefinite object pronoun} \\
        \nah{tē-}  & \trs{someone, people}            \\
        \nah{tla-} & \trs{something, stuff}           \\
        \end{tabular}
    \end{block}
    \begin{block}{Stem reduplication}
        The first syllable of a verb stem is reduplicated with:
        \begin{itemize}
        \item \nah{C\=V-} to make a frequentative,
        \item \nah{CVh-} to make a distributive.
        \end{itemize}
    \end{block}
    \begin{example}
        \begin{itemize}
        \item \nah{chōca} \trs{cries}: \nah{chōchōca} \trs{cries regularly}
        \item \nah{pāqui} \trs{is glad}: \nah{pahpāqui} \trs{is glad (on various occasions)}
        \end{itemize}
    \end{example}
    \end{column}
    \begin{column}{.18\linewidth}
    \begin{block}{Verb classes}
        \begin{tabular}{ll}
        Cl 1 & Vbs in \nah{-VCCV}           \\
        Cl 2 & Vbs in \nah{-VCV}            \\
        Cl 3 & Vbs in \nah{-VV}             \\
        Cl 4 & One-syllable Vbs in \nah{-a}
        \end{tabular}\\
        Exceptions
        \begin{itemize}
        \item Vbs in \nah{-Co}, \nah{-tla}, \nah{-ca}: Cl 1
        \item Intransitive Vbs in \nah{-hua}: Cl 1
        \item One-syllable Vbs in \nah{-i}: Cl 1
        \item Transitive Vbs in \nah{-hua}: Cl 2
        \item Vbs in \nah{-ya}: Cl 1 or 2
        \item \nah{tōna} \trs{be warm}, \nah{pāca} \trs{wash}: Cl 1
        \item \nah{zōma} \trs{become angry}: Cl 4
        \end{itemize}
    \end{block}
    \begin{block}{Verbal bases}
        B~1 is the dictionary form. Other bases are derived as follows:
        \begin{tabular}{lll}
                & B~2           & B~3                \\
        Cl 1 & B~2 = B~1     & B~3 = B~1          \\
        Cl 2 & \nah{-V > -ø} & B~3 = B~1          \\
        Cl 3 & \nah{-V > -h} & \nah{-V₁V₂ > -V₁ː} \\
        Cl 4 & \nah{-h}      & \nah{-V > -Vː}
        \end{tabular}
    \end{block}
    \begin{example}
        \begin{tabular}{lllll}
                & B~1          & B~2          & B~3          &             \\
        Cl 1 & \nah{chōca-} & \nah{chōca-} & \nah{chōca-} & \trs{cry}   \\
        Cl 2 & \nah{yōli-}  & \nah{yōl-}   & \nah{yōli-}  & \trs{live}  \\
        Cl 3 & \nah{āltia-} & \nah{āltih-} & \nah{āltī-}  & \trs{bathe} \\
        Cl 4 & \nah{cua-}   & \nah{cuah-}  & \nah{cuā-}   & \trs{eat}
        \end{tabular}
    \end{example}
    \begin{block}{Tense/Mood endings}
        \begin{threeparttable}
        \begin{tabular}{llll}
            Tense/Mood  & B & sg.                & pl.                    \\
            Present     & 1 & \nah{-ø}           & \nah{-h}               \\
            Habitual    & 1 & \nah{-ni}\tnote{1} & \nah{-nih}\tnote{1}    \\
            Imperfect   & 1 & \nah{-ya}\tnote{2} & \nah{-yah}             \\
            Preterite   & 2 & \nah{-c/ø}\tnote{3}  & \nah{-queh}            \\
            Pluperfect  & 2 & \nah{-ca}          & \nah{-cah}             \\
            Admonitive  & 2 & \nah{-h}\tnote{3}  & \nah{-(h)tin}\tnote{3} \\
            Future      & 3 & \nah{-z}           & \nah{-zqueh}           \\
            Optative    & 3 & \nah{-ø}           & \nah{-cān}             \\
            Conditional & 3 & \nah{-zquiya}      & \nah{-zquiyah}
        \end{tabular}
        \begin{tablenotes}
            \item[1] Preceding V lengthened
            \item[2] Preceding V lengthened except Cl 1
            \item[3] Cl 1: \nah{-c}, otherwise \nah{-ø}
        \end{tablenotes}
        \end{threeparttable}
    \end{block}
    \end{column}
    \begin{column}{.18\linewidth}
    \begin{block}{Irregular verbs}
        \begin{tabular}[t]{lll}
        \multicolumn{3}{l}{\nah{cā/ye} \trs{be}} \\
                & sg.         & pl.                \\
        Pres. & \nah{cah}   & \nah{cateh}        \\
        Impf. & \nah{yeya}  & \nah{yeyah}        \\
        Pret. & \nah{catca} & \nah{catcah}       \\
        Fut.  & \nah{yez}   & \nah{yezqueh}      \\
        \end{tabular}%
        \\[1ex]
        \begin{tabular}[t]{lll}
        \multicolumn{3}{l}{\nah{huītza} \trs{go}} \\
                & sg.          & pl.                \\
        Pres. & \nah{huītz}  & \nah{huītzeh}      \\
        Impf. & \nah{huītza} & \nah{huītzah}      \\
        \end{tabular}%
        \\[1ex]
        \begin{tabular}[t]{lll}
        \multicolumn{3}{l}{\nah{yā/huih} \trs{come}} \\
                & sg.        & pl.                     \\
        Pres. & \nah{yauh} & \nah{huih}              \\
        Impf. & \nah{yāya} & \nah{yāyah}             \\
        Pret. & \nah{yah}  & \nah{yahqueh}           \\
        Fut.  & \nah{yāz}  & \nah{yāzqueh}           \\
        \end{tabular}%
        \\[1ex]
        \begin{tabular}[t]{lll}
        \multicolumn{3}{l}{\nah{huāllā/huālhuih} \trs{come}} \\
                & sg.             & pl.                        \\
        Pres. & \nah{huāllauh}  & \nah{huālhuih}             \\
        Impf. & \nah{huālhuiya} & \nah{huālhuiyah}           \\
        Pret. & \nah{huāllah}   & \nah{huāllahqueh}          \\
        Fut.  & \nah{huāllaz}   & \nah{huāllazqueh}          \\
        \end{tabular}%
    \end{block}
    \end{column}
    \begin{column}{.18\linewidth}
    \begin{block}{Auxiliary verbs}
        Auxiliary verbs specify physical position of the main verb, condition under which the main verb takes place and mark aspect, and are made as follows:
        \begin{itemize}
        \item The main verb in the preterite stem
        \item Ligature morpheme \nah{-t(i)-}
        \item The auxiliary verb, bearing tense and number
        \end{itemize}
        \begin{threeparttable}
        \begin{tabular}{lll}
            \multicolumn{1}{c}{Verb} & \multicolumn{2}{c}{Meaning}                            \\
                                     & \multicolumn{1}{c}{Ind.}    & \multicolumn{1}{c}{Aux.} \\
            \nah{cah}                & \trs{be}                    & \trs{be Vb-ing}          \\
            \nah{ēhua}               & \trs{rise}                  & \trs{depart   Vb-ing}    \\
            \nah{huetzi}             & \trs{fall}                  & \trs{Vb   quickly}       \\
            \nah{huītz}              & \trs{come}                  & \trs{come   Vb-ing}      \\
            \nah{ihcac}              & \trs{stand}                 & \trs{stand   Vb-ing}     \\
            \nah{mani}               & \trs{be, cover}             & \trs{be Vb-ing}          \\
            \nah{nemi}               & \trs{live}                  & \trs{go   about Vb-ing}  \\
            \nah{(on)oc}             & \trs{lie}                   & \trs{lie   Vb-ing}       \\
            \nah{quīza}              & \trs{emerge}                & \trs{pass   Vb-ing}      \\
            \nah{yauh}\tnote{1}      & \trs{go}                    & \trs{be Vb-ing}          \\
        \end{tabular}
        \begin{tablenotes}
            \item[1] \nah{ti+yauh} > \nah{-tiuh}
        \end{tablenotes}
        \end{threeparttable}

    \end{block}
    \begin{block}{Verbs of purposive motion}
        Purposive motion suffixes take the present stem
        \begin{tabular}{llll}
                                        &     & \trs{Come}     & \trs{Go}          \\
        \multirow{2}{*}{Pres./Pret.} & sg. & \nah{-co}      & \nah{-to}         \\
                                        & pl. & \nah{-coh}     & \nah{-toh}        \\
        \multirow{2}{*}{Future}      & sg. & \nah{-quiuh}   & \nah{-tīuh}       \\
                                        & pl. & \nah{-quihuih} & \nah{-tīhuih}     \\
        \multirow{2}{*}{Optative}    & sg. & \nah{-qui}     & \nah{-h,   -ti}   \\
                                        & pl. & \nah{-quih}    & \nah{-tih,   tin} \\
        \end{tabular}
    \end{block}
    \begin{block}{Verb suffixes}
        \begin{tabular}[t]{ll}
        \multicolumn{2}{l}{Causative}     \\
        \nah{-(l)tia} & \trs{cause to Vb}
        \end{tabular}
        \\[1ex]
        \begin{tabular}[t]{ll}
        \multicolumn{2}{l}{Passive}                                  \\
        \nah{-(l)o}   & \trs{\hspace{-1em}\rdelim\}{3}{*}[be Vb'ed]} \\
        \nah{-(o)hua} &                                              \\
        \nah{-hualo}  &                                              \\
        \end{tabular}
        \\[1ex]
        \begin{tabular}[t]{ll}
        \multicolumn{2}{l}{Applicative}         \\
        \nah{-i(l)ia} & \trs{do Vb for someone}
        \end{tabular}
        \\[1ex]
        A reflexive prefix combined with either a causative or an applicative suffix creates a reverential verb
    \end{block}

    \end{column}
    \begin{column}{.18\linewidth}
    \begin{example}
        \nah{niquittato} \trs{I went to see it}

    \end{example}
    \begin{block}{Literature consulted}
        \nocite{lockhart_NahuatlWrittenLessons2001,andrews_ClaNahuatl03,jordan_BriefNotesNahuatl,canger_ReduplicationNahuatlDialectal1981,wood_OnlineNahuatlDictionary}
        \printbibliography
    \end{block}
    \end{column}
\end{columns}
\vfill
\end{frame}
